\documentclass[a4paper,11pt]{article}

\usepackage[utf8]{inputenc}
\usepackage[italian]{babel}
\usepackage[sfdefault]{roboto}
\usepackage{layaureo} % reduce margins
\usepackage{url}
\usepackage{hyperref}
\usepackage{graphicx}
\usepackage{float}
\usepackage{alertmessage}

% Setup
\hypersetup{hidelinks}
\hyphenation{knolleary}
\hyphenation{crankyoldgit}

% Document
\begin{document}
\title{IR Extender}
\author{Davide Pizzoli \\ Stefano Zenaro}
\date{Aprile 2022}

\pagenumbering{gobble}

% Title
\begin{titlepage}
  \maketitle
  \thispagestyle{empty}
\end{titlepage}

% Table of contents
\tableofcontents
\addcontentsline{toc}{section}{Indice}
\clearpage

\pagenumbering{arabic}

% Contents
\section{Descrizione}

    L'IR extender permette di controllare, anche da remoto, dispositivi infrarossi a lunga distanza attraverso Internet.

    \begin{figure}[H]
      \centering
      \includegraphics[width=\textwidth,height=\textheight,keepaspectratio]{assets/ir_extender}
      \caption{IR Extender}
    \end{figure}

    Il sistema e' composto da 3 componenti:
    \begin{itemize}
        \item Ricevitore: riceve un segnale infrarossi e lo comunica al broker MQTT attraverso un messaggio
        \item Broker MQTT: inoltra il messaggio ricevuto e lo trasmette al trasmettitore
        \item Trasmettitore: riceve il messaggio dal broker MQTT e propaga il segnale infrarossi verso il dispositivo da controllare
    \end{itemize}

    \alertinfo{Ricevitore e trasmettitore possono essere collegati alla stessa rete \mbox{Wi-Fi} oppure a due reti \mbox{Wi-Fi} differenti.}

\section{Configurazione}

    \subsection{Prerequisiti}

    Componenti elettronici:
    \begin{itemize}
      \item IR Receiver
      \item IR Transmitter
      \item ESP32 * 2
      \item Jumpers * 6
    \end{itemize}

    \noindent
    Servizi cloud:
    \begin{itemize}
      \item Broker MQTT (es. \href{https://www.hivemq.com/}{HiveMQ})
    \end{itemize}

    \noindent
    Librerie:
    \begin{itemize}
        \item Espressif 32: supporto alla board ESP32.
        \item PubSubClient (knolleary): libreria per la comunicazione attraverso il protocollo MQTT
        \item IRremoteESP8266 (crankyoldgit): libreria per inviare/ricevere segnali infrarossi
    \end{itemize}

    \subsection{Broker MQTT (HiveMQ)}
    \label{subsec:Broker}

    Per permettere la comunicazione tra ricevitore e trasmettitore e' prima necessario configurare il broker MQTT:

    \begin{enumerate}
      \item \href{https://console.hivemq.cloud/}{Registrarsi su HiveMQ}
      \item Creare un nuovo \emph{Cluster}
        \begin{figure}[H]
          \centering
          \includegraphics[width=0.8\textwidth,height=\textheight,keepaspectratio]{assets/hivemq_clusters}
        \end{figure}

        Salvarsi l'\emph{URL} del Cluster, che sarà utilizzato per configurare il Ricevitore e il Trasmettitore.

      \item Cliccare su \emph{Manage cluster}
        \begin{figure}[H]
          \centering
          \includegraphics[width=0.8\textwidth,height=\textheight,keepaspectratio]{assets/hivemq_clusterdetails}
        \end{figure}

      \item Nella sezione \emph{Access Management} aggiungere delle credenziali di accesso al \emph{Cluster}
        \begin{figure}[H]
          \centering
          \includegraphics[width=0.8\textwidth,height=\textheight,keepaspectratio]{assets/hivemq_access_management}
        \end{figure}

    \end{enumerate}

    \subsection{Ricevitore}
    \label{subsec:receiver}

    Il dispositivo ricevitore si occupa di:

    \begin{enumerate}
        \item Ricevere un segnale infrarossi da un telecomando
        \item Inviare l'informazione del segnale infrarossi, utilizzando la rete Wi-Fi, con il protocollo MQTT al Broker MQTT
    \end{enumerate}

    Per predisporre il ricevitore a ricevere segnali infrarossi e ad inoltrarli al broker MQTT occorre:
    \begin{itemize}
        \item Collegare il ricevitore infrarossi all'ESP32 come mostrato nella figura~\ref{fig:circuito_ricevitore}.
        \item Aprire la cartella \texttt{receiver/src}
        \item Rinominare il file \texttt{secrets.h.sample} in \texttt{secrets.h} e aprirlo
        \item Inserire i dati di accesso della rete WiFi e del \hyperref[subsec:Broker]{Broker MQTT}
        \item Programmare l'ESP 32
    \end{itemize}

    \begin{figure}[H]
      \centering
      \includegraphics[width=0.5\textwidth,height=\textheight,keepaspectratio]{assets/receiver_fritzing}
      \caption{Circuito Ricevitore}
      \label{fig:circuito_ricevitore}
    \end{figure}


    \subsection{Trasmettitore}
    \label{subsec:transmitter}

    Il dispositivo trasmettitore si occupa di:

    \begin{enumerate}
        \item Ricevere l'informazione dal Broker MQTT
        \item Inviare il segnale ricevuto tramite il trasmettitore infrarossi
    \end{enumerate}

    Per predisporre il trasmettitore a ricevere messaggi dal broker MQTT e di propagare segnali infrarossi occorre:

    \begin{itemize}
        \item Collegare il ricevitore infrarossi all'ESP32 come mostrato nella figura~\ref{fig:circuito_trasmettitore}.
            
        \item Aprire la cartella \texttt{transmitter/src}
        \item Rinominare il file \texttt{secrets.h.sample} in \texttt{secrets.h} e aprirlo
        \item Inserire i dati di accesso della rete WiFi e del \hyperref[subsec:Broker]{Broker MQTT}
        \item Programmare l'ESP 32
    \end{itemize}

    \begin{figure}[H]
        \centering
        \includegraphics[width=0.5\textwidth,height=\textheight,keepaspectratio]{assets/transmitter_fritzing}
        \caption{Circuito Trasmettitore}
        \label{fig:circuito_trasmettitore}
    \end{figure}

\section{Scelte progettuali}

    \subsection{Protocollo messaggio MQTT}

    La libreria che interpreta i dati rilevati dal sensore infrarossi puo' rilevare 3 tipi di protocolli:
    \begin{itemize}
        \item Protocollo a stati: composto da piu' di un valore. Usato, ad esempio, per i condizionatori.
        \item Protocollo semplice: formato da un singolo valore.
        \item Protocollo sconosciuto: non e' stato possibile rilevare il protocollo. Verranno inviati tutti i valori rilevati durante la lettura.
    \end{itemize}

    Per gestire le 3 casistiche si e' realizzato un protocollo per il messaggio MQTT.

    Per i protocolli a stati verra' inviata una stringa con il formato:

    \begin{verbatim}
<protocol>|<length>|<value1>,<value2>,...|<size>
    \end{verbatim}
            
    Dove:
    \begin{itemize}
        \item "protocol" e' un identificativo del protocollo rilevato dal sensore IR
        \item "length" e' il numero di valori "valueX"
        \item "valueX" sono valori raw letti dal sensore IR
        \item "size" (in bit) verra' usato dal metodo send() con il trasmettitore IR
    \end{itemize}

    Per i protocolli semplici verra' inviata una stringa con il formato:

    \begin{verbatim}
<protocol>|1|<value>|<size>
    \end{verbatim}

    Dove:
    \begin{itemize}
        \item "protocol" e' un identificativo del protocollo rilevato dal sensore IR
        \item "1" indica che c'e' un solo valore letto
        \item "value" e' il valore letto. Verra' usato dal metodo send() con il trasmettitore IR
        \item "size" (in bit) verra' usato dal metodo send() con il trasmettitore IR
    \end{itemize}

    Ed infine per i protocolli non riconosciuti verra' inviato un messaggio con questo formato:

    \begin{verbatim}
-1|<length>|<value1>,<value2>,...|-1
    \end{verbatim}

    Dove:
    \begin{itemize}
        \item "-1" indica che il protocollo non e' stato riconosciuto
        \item "length" e' il numero di valori "valueX"
        \item "valueX" sono valori raw letti dal sensore IR. Questi valori verranno trasmessi dal metodo sendRaw() con il trasmettitore IR
        \item "-1" (size) indica "non applicabile" poiche' il protocollo non richiede state[]
    \end{itemize}

\section{Troubleshooting}

    \subsection{Segnale infrarossi non rilevato}

    Per non rilevare troppo rumore e' stata modificata la dimensione del buffer di memoria usato per memorizzare i dati letti dal sensore infrarossi.

    E' possibile cambiare la dimensione del buffer dal file \texttt{receiver/src/main.hpp} modificando la variabile "BUFFER\_SIZE": riducendo il valore aumentera' la velocita' di lettura 
    del sensore infrarossi permettendo di leggere (dal sensore) e trasmettere (al broker MQTT) piu' valori ma aumentando la probabilita' di rilevare rumore e quindi di ottenere dati inutilizzabili dal trasmettitore.
    Aumentando la dimensione del buffer rallentera' sensibilmente la velocita' del sistema con il rischio che il sensore infrarossi non riuscira' a rilevare dati sufficenti per poterli spedire al broker MQTT.

\clearpage

% List of figures
\listoffigures
\addcontentsline{toc}{section}{\listfigurename}

\end{document}
